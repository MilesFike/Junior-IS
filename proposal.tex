% This is a template for your written document.
%
% To compile using latexmk on the command line, run the following: 
% latexmk -pdf main.tex

\documentclass[12pt]{article}
\usepackage{setspace}
\singlespace
\usepackage[left=1in,right=1in,top=1in,bottom=1in]{geometry}

\title{\textbf{Project Topic}}
\author{Miles Fike}

\begin{document}

\maketitle

In your proposal, you will provide justification on why your project matters based on work which has been done in this area, using in-line citations

I seek to create a series of interconnected computer vision AI systems that first identify a type of handwriting then properly convert the handwriting to digital text by utilizing Anthropic’s recently released Model Context Protocol. In the past, systems have been used to analyze medieval texts which are often very difficult for modern readers, and currently, many different methods for converting handwritten text utilizing AI computer vision systems exist, but many of these systems are either focused on individual styles or struggle with specific situations like ligatures. Because of the extent of previous work done in this field, I will be able to find diverse preexisting data sets with which I can train my AI systems. A user interface will be created to access the AI system required to provide context for accessing the MCP tools. Through the implementation of MCP, more specific AI models can be chosen to convert different fonts and handwriting systems to digital text rather than having one large system for all types of text.
\newpage
\section*{Appendix}
A concise list of features / user stories in the order in which they will be built. A few examples are below to demonstrate the expected scope and level of granularity; you will have more features than this.
\begin{itemize}
	\item Computer vision systems return digital text from images of handwritten text.
	\item User can view the extracted text.
	\item MCP determines the best model for text extraction based on the image provided.
	\item User can select a specific model for text extraction.
	\item System provides feedback on the accuracy of the text extraction.
	\item Drop box to upload an image for text extraction.
	\item A button downloads the text into a text file.
	\item A text box displays the tool used to read the text to clarify to the user what model was used.

\end{itemize}


\bibliographystyle{acm}
\bibliography{bibliography.bib}

\end{document}
