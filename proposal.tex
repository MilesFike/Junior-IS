% This is a template for your written document.
%
% To compile using latexmk on the command line, run the following: 
% latexmk -pdf main.tex

\documentclass[12pt]{article}
\usepackage{setspace}
\singlespace
\usepackage[left=1in,right=1in,top=1in,bottom=1in]{geometry}

\title{\textbf{Project Topic}}
\author{Miles Fike}

\begin{document}

\maketitle

In your proposal, you will provide justification on why your project matters based on work which has been done in this area, using in-line citations

In the past, many different computer visions systems have been created to convert handwritten or printed text of different fonts and styles to computer readable text. These systems all tend to involve similar principles of computer vision to decipher text using systems trained on data sets of images of text. More recently Model Context Protocol (MCP) has been released. As is explained on the Model Context Protocol Website, this technology assists AI systems use of technologies by providing connections between AI systems and tools, workflows, and data sources that they need \cite{modelContextProtocol/getting-started/intro}. I intend to approach issues and limitations of transcription technologies through the implementation of MCP, more specific AI models can be chosen to convert different fonts and handwriting systems to digital text rather than having one large system for all types of text. 
\newpage
\section*{Appendix}
A concise list of features / user stories in the order in which they will be built. A few examples are below to demonstrate the expected scope and level of granularity; you will have more features than this.
\begin{itemize}
	\item Computer vision systems return digital text from images of handwritten text.
	\item User can view the extracted text.
	\item MCP determines the best model for text extraction based on the image provided.
	\item User can select a specific model for text extraction.
	\item System provides feedback on the accuracy of the text extraction.
	\item Drop box to upload an image for text extraction.
	\item A button downloads the text into a text file.
	\item A text box displays the tool used to read the text to clarify to the user what model was used.

\end{itemize}


\bibliographystyle{acm}
\bibliography{bibliography.bib}

\end{document}
